\documentclass[a4paper]{ctexart}

\usepackage{tabularx}

\usepackage{fancyhdr}
\pagestyle{fancy}  % configure page style using the fancyhdr package
\fancyhf{}  % clear existing header/footer

\begin{document}

\begin{center}
{\Large
网安2024春计算机组成原理与汇编语言:第03次作业
}
\end{center}

\begin{flushleft}
\begin{tabularx}{0.5\textwidth}{ >{\raggedright}X >{\raggedright}X }
姓名&学号\\
\end{tabularx}
\end{flushleft}

% 细节上有改动

% 6E3.1 (5E3.1)
What is \verb|5ED4| minus \verb|07A4| when these values represent unsigned 16-bit hexadecimal numbers?
The result should be written in hexadecimal. Show your work.

% 6E3.2 (5E3.2)
What is \verb|5ED4| minus \verb|07A4| when these values represent signed 16-bit hexadecimal numbers
stored in sign-magnitude format? The result should be written in hexadecimal. Show your work.

% 6E3.7 (5E3.7)
Assume 185 and 122 are signed 8-bit decimal integers stored in sign-magnitude format.
Calculate 185 + 122.
Is there overflow, underflow, or neither?

% 6E3.15 (5E3.15)
Calculate the time necessary to perform a multiplication using the approach described in the text
(31 adders stacked vertically)
if an integer is 8 bits wide and an adder takes 4 time units.

% 6E3.16 (5E3.16)
Calculate the time necessary to perform a multiplication using the approach given in Figure 3.7
if an integer is 8 bits wide and an adder takes 4 time units.

% 6E3.20 (5E3.20)
What decimal number does the bit pattern \verb|0x0c00 0000| represent
if it is a two's complement integer? An unsigned integer?

% 6E3.22 (5E3.22)
What decimal number does the bit pattern \verb|0x0c00 0000| represent
if it is a floating point number?
Use the IEEE 754 standard.

% 6E3.23 (5E3.23)
Write down the binary representation of the decimal number 63.25
assuming the IEEE 754 single precision format.

% 6E3.27 (5E3.27)
IEEE 754-2008 contains a half precision that is only 16 bits wide.
The left most bit is still the sign bit, the exponent is 5 bits wide and has a bias of 15,
and the mantissa is 10 bits long.
A hidden 1 is assumed.
Write down the bit pattern to represent $-1.5625 \times {10}^{-1}$ assuming a version of this format,
which uses an excess-16 format to store the exponent.
Comment on how the \textbf{range} and \textbf{accuracy} of this 16-bit floating point format
compares to the single precision IEEE 754 standard.

% 6E3.29 (5E3.29)
Calculate the sum of $2.6125 \times {10}^{1}$ and $4.150390625 \times {10}^{-1}$ by hand,
assuming both numbers are stored in the 16-bit half precision described in Exercise 3.27.
Assume 1 guard bit, 1 round bit, and 1 sticky bit, and round to the nearest even.
\textbf{Show all the steps.}

% 6E3.30 (5E3.30)
Calculate the product of $-8.0546875 \times {10}^{0}$ and $1.79931640625 \times {10}^{-1}$ by hand,
assuming both numbers are stored in the 16-bit half precision format described in Exercise 3.27.
Assume 1 guard bit, 1 round bit, and 1 sticky bit, and round to the nearest even.
Show all the steps;
however, as is done in the example in the text, you can do the multiplication in human-readable format instead of using the techniques described in Exercises 3.12 through 3.14.
Indicate if there is overflow or underflow.
Write your answer in both the 16-bit floating point format described in Exercise 3.27
and also as a decimal number.
How accurate is your result?
How does it compare to the number you get if you do the multiplication on a calculator?

\end{document}
